\documentclass[english,10pt,table]{beamer}
%\documentclass[english,10pt,handout]{beamer}
%\documentclass[slidestop,usepdftitle=false]{beamer}

\input{aa-style.tex}
\lecture[Clustering]{K-means based method}{lecture-text}

%\date[]{~~}
%%%%%%%%%%%%%%%%%%%%%%%%%%%%%%%%%%%%%%%%%%%%%%%%%%%%%%%%%%%%%%%%%%%%%
%%%%%%%%%%%%%%%%%%%%%%%%%%%%%%%%%%%%%%%%%%%%%%%%%%%%%%%%%%%%%%%%%%%%%
\begin{document}

\begin{frame}
  \maketitle
\end{frame}

\section*{Nội dung}
	\begin{frame}\frametitle<presentation>{Nội dung}
  	\tableofcontents
	\end{frame}


%%%%%%%%%%%%%%%%%%%%%%%%%%%
\section{Giới thiệu về phân cụm}
\subsection{Phân cụm}
\begin{frame}{Phân cụm}		
	\begin{block}{Phân cụm là gì ?}
		\begin{itemize}\small
	 	\item Việc tổ chức dữ liệu chưa được gán nhãn (label) vào các nhóm tương tự nhau được gọi là phân cụm \\
	 	\item Một cụm là một tập hợp các phần tử dữ liệu có sự giống nhau về mặt dữ liệu và sẽ khác với các phần tử dữ liệu ở các cụm khác.
		\end{itemize}
	 
  \end{block}
\includegraphics[scale=0.5]{images/clusters.png}
\end{frame}

%%%%%%%%%%%%%%%%%%%%%%%%%%%%
\subsection{Kỹ thuật phân cụm}
\begin{frame}{Kỹ thuật phân cụm}	
	\includegraphics[scale=0.5]{images/cluster-tech.png}
\end{frame}

%%%%%%%%%%%%%%%%%%%%%%%%%%%%
\section{Vấn đề \& K-means}
\subsection{K-means}
\begin{frame}{Vấn đề}
	\begin{block} {Khi nào nghĩ đến K-means}
		\begin{itemize} \small
			\item Không biết nhãn(label) của điểm dữ liệu
			\item Mục đích : Phân dữ liệu thành các cụm (cluster) khác nhau sao cho dữ liệu trong một cụm có tính chất giống nhau.
		\end{itemize}
	\end{block}

	\begin{block}  {K-means cluster}
		\begin{itemize} \small
			\item K-means được đề xuất bởi MacQueen năm 1967
			\item Giải thuật k-means chia tập dữ liệu thành k cụm (cluster)
			\begin{itemize}
				\item Mỗi cluster có một điểm trung tâm , gọi là \alert{$centroid$}
				\item K được chỉ định bởi nhân viên phân tích dữ liệu (Data analytics)
			\end{itemize}
		\end{itemize}
	\end{block}

\end{frame}
%%%%%%%%%%%%%%%%%%%%%%%%%%%%
\begin{frame} {Giải thật k-means}
	\begin{block}{Nhập giá trị k, giải thuật k-means sẽ thực thi các bước như sau:}
		\begin{enumerate}
				\item Chọn ngẫu nhiên k điểm dữ liệu làm \alert{$centroids$}, điểm trung tâm của cụm dữ liệu.
				\item Phân mỗi điểm dữ liệu vào cluster có điểm trung tâm (center) gần nó nhất.
				\item Nếu việc gán dữ liệu vào từng cluster ở bước 2 không thay đổi so với vòng lặp trước nó thì ta dừng thuật toán.
				\item Cập nhật center cho từng cluster bằng cách lấy lấy trung bình cộng của tất cả các điểm dữ liệu đã được gán vào cluster đó sau bước 2.
				\item Quay lại bước 2.
		\end{enumerate}
	\end{block}
\end{frame}

%%%%%%%%%%%%%%%%%%%%%%%%%%%%

\begin{frame}
	\begin{block}{Cơ chế của giải thuật K-means có thể tổng quát bằng sơ đồ dưới đây:}
		\begin{center}
			\includegraphics[scale=0.4]{images/flowchart.png}
		\end{center}
	\end{block}
\end{frame}

%%%%%%%%%%%%%%%%%%%%%%%%%%%%

\begin{frame} {Cách tính khoảng cách và điểm trung tâm}
	\begin{block} {Cách tính khoảng cách trong giải thuật K-means}
		% cach tinh khoang cach
		Để tìm được điểm dữ liệu gần với điểm trung tâm nào nhất thì ta dựa vào giá trị nhỏ nhất của hàm tính khoảng cách Euclidean.
		\begin{itemize}
			\item Với dữ liệu 1 chiều, với 2 điểm dữ liệu p và q \\ $\sqrt{p-q}=|p-q|$
			\item Với dữ liệu 2 chiều, với 2 điểm dữ liệu p (p1,p2) và q (q1, q2) \\ $\sqrt{(p1-q1)^2 + (p2-q2)^2}$
			\item Với dữ liệu 3 chiều, với 2 dữ liệu p(p1,p2,p3) và q(q1,q2,q3)\\ $\sqrt{(p1-q1)^2 + (p2-q2)^2 + (p3-q3)^2}$
			\item Với dữ liệu n chiều, với 2 dữ liệu p(p1,p2,...,pn) và q(q1,q2,...,qn) \\ $\sqrt{(p1-q1)^2 + (p2-q2)^2 + ... + (pn-qn)^2}$
		\end{itemize}
	\end{block}
	\begin{block} {Cách tính điểm trung tâm}
		Điểm trung tâm \alert{$centroid$} của cluster là trung bình cộng của các điểm trong cluster đó.
	\end{block}
\end{frame}

%%%%%%%%%%%%%%%%%%%%%%%%%%%%
\subsection{Mô phỏng K-means}
\begin{frame} {Mô phỏng K-means}
\begin{block}{Chọn ngẫu nhiên các centroid}
	\begin{center}
		\includegraphics[scale=0.4]{images/step1.png}
	\end{center}
\end{block}
\end{frame}

%%%%%%%%%%%%%%%%%%%%%%%%%%%%
\begin{frame} {Mô phỏng K-means}
\begin{block}{Xác định cluster cho từng điểm dữ liệu}
	\begin{center}
		\includegraphics[scale=0.4]{images/step2.png}
	\end{center}
\end{block}
\end{frame}

%%%%%%%%%%%%%%%%%%%%%%%%%%%%
\begin{frame} {Mô phỏng K-means}
\begin{block}{Xác định lại centroid cho các cluster}
	\begin{center}
		\includegraphics[scale=0.4]{images/step3.png}
	\end{center}
\end{block}
\end{frame}
% Mô phỏng giải thuật k-means bằng hình ảnh

%%%%%%%%%%%%%%%%%%%%%%%%%%%%
\begin{frame} {Mô phỏng K-means}
\begin{block}{Kết quả của vòng lặp thứ nhất}
	\begin{center}
		\includegraphics[scale=0.4]{images/result1.png}
	\end{center}
\end{block}
\end{frame}

%%%%%%%%%%%%%%%%%%%%%%%%%%%%
\begin{frame} {Mô phỏng K-means}
\begin{block}{Vòng lặp thứ 2, xác định lại các điểm centroid}
	\begin{center}
		\includegraphics[scale=0.4]{images/ite1.png}
	\end{center}
\end{block}
\end{frame}

%%%%%%%%%%%%%%%%%%%%%%%%%%%%
\begin{frame} {Mô phỏng K-means}
\begin{block}{Kết quả của vòng lặp thứ hai}
	\begin{center}
		\includegraphics[scale=0.4]{images/result2.png}
	\end{center}
\end{block}
\end{frame}

%%%%%%%%%%%%%%%%%%%%%%%%%%%%
%\section{Ứng dụng}
%\begin{frame}{Ứng dụng phân loại hình ảnh}
%\begin{block} {Examples of visual words}
%	\begin{center}
%		\includegraphics[scale=0.4]{images/demo1.png}
%	\end{center}
%\end{block}
%\end{frame}

%%%%%%%%%%%%%%%%%%%%%%%%%%%%
\section{Điểm mạnh và điểm yếu}
\subsection{Điểm mạnh}
\begin{frame}{Điểm mạnh và điểm yếu}
\begin{block} {Điểm mạnh}
	\begin{itemize}
		\item Đơn giản: Dễ dàng hiểu và thực thi
		\item Rất hiệu quả: Độ phức tạp chỉ là O(tkn), trong đó: 
			\begin{itemize}
				\item n là số lượng điểm dữ liệu (data point)
				\item k là số cụm (cluster)
				\item t là số lần lặp cho đến khi hội tụ
			\end{itemize}
		Vì k và t đều nhỏ nên giải thuật K-mean là thuật toán có độ phức tạp tuyến tính
	\end{itemize}
	
\end{block}
	k-means là giải thuật phổ biến.
\end{frame}

%%%%%%%%%%%%%%%%%%%%%%%%%%%%
\subsection{Điểm yếu}
\begin{frame}{Điểm mạnh và điểm yếu}
\begin{block} {Điểm yếu}
	\begin{itemize}
		\item Thuật toán chỉ áp dụng được khi các giá trị trung bình được xác định.
		\item Phải xác định số lượng cụm (cluster)
		\item Giải thuật bị ảnh hưởng bởi các điểm \alert{$outliers$}
			\begin{itemize}
			\item Outliers là những điểm dữ liệu ở quá xa so với các điểm dữ liệu khác.
			\item Outliers tồn tại do lỗi trong quá trình ghi dữ liệu hoặc một số điểm dữ liệu đặc biệt có các giá trị rất khác nhau.
			\end{itemize}
	\end{itemize}
\end{block}
\end{frame}

%%%%%%%%%%%%%%%%%%%%%%%%%%%%
\begin{frame}{Điểm yếu}
	\begin{block} {Outliers}
	\begin{center}
		\includegraphics[scale=0.4]{images/outlier.png}
	\end{center}
	\end{block}
\end{frame}


%%%%%%%%%%%%%%%%%%%%%%%%%%%%
\begin{frame}{Giải quyết Outlier}
\begin{itemize}
	\item Xóa các điểm dữ liệu ở xa centroid so với các điểm dữ liệu khác.
	 	\begin{itemize}
	 		\item Để đảm bảo an toàn, chúng ta có thể theo dõi các điểm này qua vài lần lặp trước khi quyết loại bỏ.
	 	\end{itemize}
	\item Một cách khác, bằng cách lấy mẫu ngẫu nhiên: Vì trong khi lấy mẫu, chúng ta chỉ chọn tập hợp con của các điểm dữ liệu, nên rất ít khả năng chọn trúng điểm outlier.
		\begin{itemize}
			\item Gán những điểm dữ liệu còn lại cho các clusters theo khoảng cách hoặc phân lớp.
		\end{itemize}
\end{itemize}
\end{frame}

%%%%%%%%%%%%%%%%%%%%%%%%%%%%
\begin{frame}{Điểm yếu}
\begin{block} {Sensitivity to initial seeds}
	\begin{center}
		\includegraphics[scale=0.4]{images/seed.png}
	\end{center}
\end{block}
\end{frame}

%%%%%%%%%%%%%%%%%%%%%%%%%%%%
\begin{frame}{Điểm yếu}
\begin{block} {Cấu trúc dữ liệu đặc biệt}
	Thuật toán k-means không phù hợp để khai thác các cluster không phải là hyper - ellipsoids (hyper-spheres)
	\begin{center}
		\includegraphics[scale=0.4]{images/spec_data.png}
	\end{center}
\end{block}
\end{frame}

%%%%%%%%%%%%%%%%%%%%%%%%%%%%
\section{Tổng kết}
\begin{frame}{Tổng kết}
\begin{block} {Tổng kết giải thuật K-means}
	\begin{itemize}
		\item Mặc dù có điểm yếu nhưng k-means vẫn là giải thuật phổ biến do tính đơn giản và tính hiệu quả của nó.
		\item Chưa có bằng chứng nào chứng tỏ giải thuật phân cụm nào là tốt nhất.
	\end{itemize}
\end{block}
\end{frame}

\section{Tham khảo}
\begin{frame}{Tham khảo}
	\begin{enumerate}
		\item \href{http://www.mit.edu/~9.54/fall14/slides/Class13.pdf}{http://www.mit.edu/~9.54/fall14/slides/Class13.pdf}
		\item \href{https://www.saedsayad.com/clustering_kmeans.htm}{https://www.saedsayad.com/clustering\_kmeans.htm}
		\item Bing Liu, "Web data Mining", Springer, Second Edition, 2007
	\end{enumerate}
\end{frame}

\end{document}